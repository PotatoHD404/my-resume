%-------------------------
% Rezume, a latex resume template for developers
% Author : Nanu Panchamurthy
% Based off of: https://github.com/sb2nov/resume
% License : MIT

% Hope this resume template helps you land an awesome job. If you found this helpful, please consider starring the github repo here, .
%-------------------------



%------------PACKAGES----------------
\documentclass[a4paper,11pt]{article}

\usepackage{verbatim} % reimplements the "verbatim" and "verbatim*" environments

\usepackage{titlesec} % provides an interface to sectioning commands i.e. custom elements

\usepackage{color} % provides both foreground and background color management

\usepackage{enumitem} % provides control over enumerate, itemize and description

\usepackage{fancyhdr} % provides extensive facilities for constructing headers, footers and also controlling their use

\usepackage{tabularx} % defines an environment tabularx, extension of "tabular" with an extra designator x, paragraph like column whose width automatically expands to fill the width of the environment

\usepackage{latexsym} % provides mathematical symbols

\usepackage{marvosym} % provides martin vogel's symbol font which contains various symbols

\usepackage[empty]{fullpage} % sets margins to one inch and removes headers, footers etc..

\usepackage[hidelinks]{hyperref} % removes color and shadow of hyperlinks

\usepackage[normalem]{ulem} % provides "\ul" (uline) command which will break at line breaks

\usepackage[english, russian]{babel} % provides culturally determined typographical rules for wide range of languages
%-----------------------------------------

\input glyphtounicode % converts glyph names to unicode
\pdfgentounicode=1 % ensures pdfs generated are ats readable

%----------FONT OPTIONS-------------------
\usepackage[default]{sourcesanspro} % uses the font source sans pro
\urlstyle{same} % changes url font from default urlfont to font being used by the document
%-----------------------------------------


%----------MARGIN OPTIONS-----------------
\pagestyle{fancy} % set page style to one configured by fancyhdr
\fancyhf{} % clear all header and footer fields

\renewcommand{\headrulewidth}{0in} % sets thickness of linerule under header to zero
\renewcommand{\footrulewidth}{0in} % sets thickness of linerule over footer to zero

\newcommand{\Csh}{C{\lserif\#}}

\setlength{\tabcolsep}{0in} % sets thickness of column separator in tables to zero

% origin of the document is one inch from the top and from and the left
% oddsidemargin and evensidemargin both refer to the left margin
% right margin is indirectly set using oddsidemargin
\addtolength{\oddsidemargin}{-0.5in}
\addtolength{\topmargin}{-0.5in}

\addtolength{\textwidth}{1.0in} % sets width of text area in the page to one inch
\addtolength{\textheight}{1.0in} % sets height of text area in the page to one inch

\raggedbottom{} % makes all pages the height of current page, no extra vertical space added
\raggedright{} % makes all pages the width of current page, no extra horizontal space added
%------------------------------------------


%--------SECTIONING COMMANDS---------
% \titleformat{<command>}
%   [<shape>]{<format>}{<label>}{<sep>}
%     {<before-code>}[<after-code>]

% command is the sectioning command to be redefined
% shape is the style of the font; scshape stands for small caps style
% format is the format to be applied to whole title- label and text; absent here
% label defines the label
% sep is the horizontal separation between label and title body
% before-code is the code to be executed before
% after-code is the code to be executed after

\titleformat{\section}
  {\scshape\large}{}
    {0em}{\color{blue}}[\color{black}\titlerule\vspace{0pt}]
%-------------------------------------


%--------REDEFINITIONS----------------
% redefines the style of the bullet point
\renewcommand\labelitemii{$\vcenter{\hbox{\tiny$\bullet$}}$}

% redefines the underline depth to 2pt
\renewcommand{\ULdepth}{2pt}
%-------------------------------------


%--------CUSTOM COMMANDS--------------
%\vspace{} defines a vertical space of given size, modifying this in custom commands can help stretch or shrink resume to remove or add content

% resumeItem renders a bullet point
\newcommand{\resumeItem}[1]{
  \item\small{#1}
}

% commands to start and end itemization of resumeItem, rightmargin set to 0.11in to avoid the overflow of resumetItem beyond whatever resumeItemHeading is being used
\newcommand{\resumeItemListStart}{\begin{itemize}[rightmargin=0.11in]}
\newcommand{\resumeItemListEnd}{\end{itemize}}

% resumeSectionType renders a bolded type to be used under a section, used as skill type here, middle element is used to keep ":"s in the same vertical line
\newcommand{\resumeSectionType}[3]{
  \item\begin{tabular*}{0.96\textwidth}[t]{
    p{0.15\linewidth}p{0.02\linewidth}p{0.81\linewidth}
  }
    \textbf{#1} & #2 & #3
  \end{tabular*}\vspace{-2pt}
}

% resumeTrioHeading renders three elements in three columns with second element being italicized and first element bolded, can be used for projects with three elements
\newcommand{\resumeTrioHeading}[3]{
  \item\small{
    \begin{tabular*}{0.96\textwidth}[t]{
      l@{\extracolsep{\fill}}c@{\extracolsep{\fill}}r
    }
      \textbf{#1} & \textit{#2} & #3
    \end{tabular*}
  }
}

% resumeQuadHeading renders four elements in a two columns with the second row being italicized and first element of first row bolded, can be used for experience and projects with four elements
\newcommand{\resumeQuadHeading}[4]{
  \item
  \begin{tabular*}{0.96\textwidth}[t]{l@{\extracolsep{\fill}}r}
    \textbf{#1} & #2 \\
    \textit{\small#3} & \textit{\small #4} \\
  \end{tabular*}
}

% resumeQuadHeadingChild renders the second row of resumeQuadHeading, can be used for experience if different roles in the same company need to added
\newcommand{\resumeQuadHeadingChild}[2]{
  \item
  \begin{tabular*}{0.96\textwidth}[t]{l@{\extracolsep{\fill}}r}
    \textbf{\small#1} & {\small#2} \\
  \end{tabular*}
}

% commands to start and end itemization of resumeQuadHeading, lefmargin for left indent of 0.15in for resumeItems
\newcommand{\resumeHeadingListStart}{
  \begin{itemize}[leftmargin=0.15in, label={}]
}
\newcommand{\resumeHeadingListEnd}{\end{itemize}}
%-------------------------------------------


%__________________RESUME____________________
% You can rearrange sections in any order you may prefer
\begin{document}

%-----------CONTACT DETAILS------------------
% Make sure all the details are correct, you can add more links in the first row of second column if needed

\begin{tabular*}{\textwidth}{l@{\extracolsep{\fill}}r}
  \textbf{\Huge Корначук Марк \vspace{2pt}} & % row = 1, col = 1
  Город: Москва \\ % row = 1, col = 2
  \href{https://github.com/PotatoHD404}{\uline{GitHub}} $|$ % row = 2, col = 1
  Email: \href{mailto:kornachuk.mark@gmail.com}{\uline{kornachuk.mark@gmail.com}} $|$ % row = 2, col = 2
  Telegram: \href{https://t.me/PotatoHD404}{\uline{@potatohd404}} \\ % row = 2, col = 2
\end{tabular*}
%--------------------------------------------


%-----------SUMMARY--------------------------
% Keep this short, simple and straigth to point

\section{Full Stack Developer}
\small{
  Я Full Stack разработчик с опытом работы в React, Typescript, \Csh{}, Python, SQL и Linux. У меня есть опыт в разработке веб-приложений, управлении базами данных и взаимодействии с API. Я постоянно самосовершенствуюсь и изучаю новые технологии и методологии, что помогает улучшать качество моей работы.
}
%--------------------------------------------


%--------------SKILLS------------------------
% Add or remove resumeSectionTypes according to your needs

\section{Навыки}
  \resumeHeadingListStart{}
    \resumeSectionType{Языки}{:}{Python, TypeScript, \Csh{}, SQL, Bash}
    \resumeSectionType{Фреймворки}{:}{React.js, Next.js, ASP.NET, Django}
    \resumeSectionType{Библиотеки}{:}{Redux, React Router}
    \resumeSectionType{Базы данных}{:}{PostgreSQL, Firebase, DynamoDB}
    \resumeSectionType{Инструменты}{:}{Git, GitHub, Terraform, Docker}
    \resumeSectionType{Облака}{:}{Yandex Cloud, AWS, Vercel}
  \resumeHeadingListEnd{}
%--------------------------------------------


%-----------EXPERIENCE-----------------------
% Distill all your talking points to small bullet points which follow the pattern "challenge-action-result" for maximum efficiency. Try to quantify (use numbers) your points whenver possible, highlist words of importance

% \section{Опыт}
% \resumeHeadingListStart{}
%   \resumeQuadHeading{Web Developer}{Apr 2022 -- Present}
%   {Anycompany}{Remote -- AnyCity, Anystate, Anycountry}
%     \resumeItemListStart{}
%       \resumeItem{Designed and developed dynamic and responsive websites using \textbf{HTML, CSS, JavaScript, and PHP}}
%       \resumeItem{Worked with \textbf{REST APIs} to retrieve and display data from databases}
%       \resumeItem{Improved \textbf{website performance} and speed through optimization techniques by \textbf{55\%}}
%     \resumeItemListEnd{}

%   \resumeQuadHeading{Backend Developer}{Aug 2021 -- Nov 2022}
%   {Anycompany}{Anycity, Anystate, Anycountry}
%     \resumeItemListStart{}
%       \resumeItem{Worked with \textbf{MVC frameworks} to develop robust and scalable backends}
%       \resumeItem{Troubleshot and \textbf{fixed bugs} and issues in the backend to ensure \textbf{smooth operation} of the applications}
%     \resumeItemListEnd{}

%   \resumeQuadHeadingChild{Backend Developer Intern}{Jan 2021 -- Aug 2021}
%     \resumeItemListStart{}
%       \resumeItem{Assisted senior web developers in the design and development of websites using \textbf{HTML, CSS, and JavaScript}}
%     \resumeItemListEnd{}
% \resumeHeadingListEnd{}
%---------------------------------------------


%-----------EDUCATION-------------------------
% Mention your CGPA, if its good, in the first row of second column

\section{Образование}
  \resumeHeadingListStart{}
    \resumeQuadHeading{Национальный Исследовательский Ядерный Университет "МИФИ"}{Бакалавариат}
    {Программная инжинерия}{Сентябрь 2020 -- Июль 2024}
  \resumeHeadingListEnd{}
%---------------------------------------------


%-----------PROJECTS--------------------------
% Use resumeQuadHeading if four elements are feasible (ex: demo video link), else use resumeTrioHeading. Keep the bullet points simple and concise and try to cover wide variety of skills you have used to build these projects

\section{Проекты}
  \resumeHeadingListStart{}
    \resumeTrioHeading{\href{https://daily-mephi.ru}{\uline{daily-mephi.ru}}}{Next.js, Typescript, Serverless, Yandex cloud}{\href{https://github.com/MEPhI-Floppas/daily-mephi}{\uline{Source Code}}}
      \resumeItemListStart{}
        \resumeItem{Разработал сдуденческий портал для отзывов о преподавателях и обмена материалами с использованием \textbf{Next.js, TypeScript, Prisma, tRPC}}
        \resumeItem{Полностью развернут в Yandex cloud при помощи terraform и yandex cloud functions (serverless)}
        \resumeItem{В проекте настроен Github Actions для непрерывной разработки и тестирования}
        \resumeItem{Для тестирования в проекте используются Jest (для интеграционных и unit-тестов) и Storybook (для тестов компонентов)}
        \resumeItem{В ходе проекта был также разработан способ загрузки файлов в notion для замены облачного хранилища}
        \resumeItem{Кроме того был разработан ORM для YDB на Node.js}
        \resumeItem{Была создана система рендеринга React компонентов в PNG}
        \resumeItem{Был создан прототип библиотеки для представления бекенда на next.js в виде, похожем на Spring (в виде классов), а также dependency injection для этой библиотеки}
        \resumeItem{Главным вызовом в проекте было развертывание в облаке, так как не существовало технологии развертывания next.js в yandex cloud}
      \resumeItemListEnd{}

      \resumeTrioHeading{\href{https://paralleldb.vercel.app/}{\uline{ParallelDB}}}{\Csh{}, ASP.Net, AWS Lambda, Vue3, Typescript}{\href{https://github.com/PotatoHD404/ParallelDB}{\uline{Source Code}}}
      \resumeItemListStart{}
        \resumeItem{Простая база данных на \Csh{}, позволяющая обрабатывать запросы параллельно}
        \resumeItem{Написан парсер для SQL, DSL для программного использования}
        \resumeItem{API на ASP.net для взаимодействия с фронтендом, захосченое на AWS lambda}
        \resumeItem{Фронтенд на Vue3 для отправки запросов к БД и последующего отображения}
        \resumeItem{Наибольшую сложность представлял парсер SQL и последующую обработку запроса}
      \resumeItemListEnd{}
    \resumeTrioHeading{\href{https://t.me/potatohd_checker_bot}{\uline{website-checker}}}{Telegram, Golang, AWS lambda, dynamodb}{\href{https://github.com/PotatoHD404/website-checker-bot}{\uline{Source Code}}}
      \resumeItemListStart{}
        \resumeItem{Телеграм бот для проверки изменений сайта, написанный на golang}
        \resumeItem{Каждые 5 минут происходит проверка изменений всех сайтов из базы данных DynamoDB}
        \resumeItem{Можно подписаться на изменения любого сайта, но только админ может добавить новый}
        \resumeItem{Основной сложностью было настроить terraform, GitHub actions и AWS SDK}
      \resumeItemListEnd{}
  \resumeHeadingListEnd{}
%--------------------------------------------


%----------------OTHERS----------------------
% You can add your acheivements, accolades, certifications etc. here.

% \section{Certifications}
%   \resumeItemListStart{}
%     \resumeItem{\href{https://dummy-certification.com}{\uline{Certified Web Developer by the W3C}}}
%     \resumeItem{\href{https://dummy-certification.com}{\uline{Microsoft Certified: Azure Developer Associate}}}
%     \resumeItem{\href{https://dummy-certification.com}{\uline{AWS Certified Developer - Associate}}}
%   \resumeItemListEnd{}
%--------------------------------------------

\end{document}